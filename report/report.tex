\documentclass{bmcart}

%%%%%%%%%%%%%%%%%%%%%%%%%%%%%%%%%%%%%%%%%%%%%%
%%                                          %%
%% CARGA DE PAQUETES DE LATEX               %%
%%                                          %%
%%%%%%%%%%%%%%%%%%%%%%%%%%%%%%%%%%%%%%%%%%%%%%

%%% Load packages
\usepackage{amsthm,amsmath}
\usepackage{graphicx}
%\RequirePackage[numbers]{natbib}
%\RequirePackage{hyperref}
\usepackage[utf8]{inputenc} %unicode support
%\usepackage[applemac]{inputenc} %applemac support if unicode package fails
%\usepackage[latin1]{inputenc} %UNIX support if unicode package fails


%%%%%%%%%%%%%%%%%%%%%%%%%%%%%%%%%%%%%%%%%%%%%%
%%                                          %%
%% COMIENZO DEL DOCUMENTO                   %%
%%                                          %%
%%%%%%%%%%%%%%%%%%%%%%%%%%%%%%%%%%%%%%%%%%%%%%

\begin{document}

	\begin{frontmatter}
	
		\begin{fmbox}
			\dochead{Research}
			
			%%%%%%%%%%%%%%%%%%%%%%%%%%%%%%%%%%%%%%%%%%%%%%
			%% INTRODUCIR TITULO PROYECTO               %%
			%%%%%%%%%%%%%%%%%%%%%%%%%%%%%%%%%%%%%%%%%%%%%%
			
			\title{Aniridia y su relación con otras patologías}
			
			%%%%%%%%%%%%%%%%%%%%%%%%%%%%%%%%%%%%%%%%%%%%%%
			%% AUTORES. METER UNA ENTRADA AUTHOR        %%
			%% POR PERSONA                              %%
			%%%%%%%%%%%%%%%%%%%%%%%%%%%%%%%%%%%%%%%%%%%%%%
			
			\author[
			  addressref={aff1},                   % ESTA LINEA SE COPIA IGUAL PARA CADA AUTOR
			  email={teresavegamar@uma.es}   % VUESTRO CORREO ACTIVO
			]{\inits{T.V.}\fnm{Teresa V.} \snm{Vega Martínez}} % inits: INICIALES DE AUTOR, fnm: NOMBRE DE AUTOR, snm: APELLIDOS DE AUTOR
			
			\author[
			addressref={aff1},                   % ESTA LINEA SE COPIA IGUAL PARA CADA AUTOR
			corref={aff1},                       % ESTA LINEA SOLO DEBE TENERLA EL COORDINADOR DEL GRUPO
			email={13albertomontero@uma.es}   % VUESTRO CORREO ACTIVO
			]{\inits{A.M.}\fnm{Alberto M.} \snm{Montero Solera}} % inits: INICIALES DE AUTOR, fnm: NOMBRE DE AUTOR, snm: APELLIDOS DE AUTOR
			
			
			%%%%%%%%%%%%%%%%%%%%%%%%%%%%%%%%%%%%%%%%%%%%%%
			%% AFILIACION. NO TOCAR                     %%
			%%%%%%%%%%%%%%%%%%%%%%%%%%%%%%%%%%%%%%%%%%%%%%
			
			\address[id=aff1]{%                           % unique id
			  \orgdiv{ETSI Informática},             % department, if any
			  \orgname{Universidad de Málaga},          % university, etc
			  \city{Málaga},                              % city
			  \cny{España}                                    % country
			}
		
		\end{fmbox}% comment this for two column layout
		
		\begin{abstractbox}
		
			\begin{abstract} % abstract
			
			%%%%%%%%%%%%%%%%%%%%%%%%%%%%%%%%%%%%%%%%%%%%%%%
			%% RESUMEN BREVE DE NO MAS DE 100 PALABRAS   %%
			%%%%%%%%%%%%%%%%%%%%%%%%%%%%%%%%%%%%%%%%%%%%%%%	
			
			\end{abstract}
			
			%%%%%%%%%%%%%%%%%%%%%%%%%%%%%%%%%%%%%%%%%%%%%%
			%% PALABRAS CLAVE DEL PROYECTO              %%
			%%%%%%%%%%%%%%%%%%%%%%%%%%%%%%%%%%%%%%%%%%%%%%
			
			\begin{keyword}
			\kwd{sample}
			\kwd{article}
			\kwd{author}
			\end{keyword}
		
		
		\end{abstractbox}
	
	\end{frontmatter}
	
	%%%%%%%%%%%%%%%%%%%%%%%%%%%%%%%%%%%%%%%%%%%%%%%%%%%%%%%%%%%%%%%%%%%%%%%%%%%%%%%%%%%%%%%%
	%% EJEMPLO DE LATEX %%                                                                %%
	%% BORRAR ANTES DE ENTREGAR!!!!!!!!!!!!!!!!!!!!!!!!!!!!!!!!!!!!!!!!!!!!!              %%
	%%%%%%%%%%%%%%%%%%%%%%%%%%%%%%%%%%%%%%%%%%%%%%%%%%%%%%%%%%%%%%%%%%%%%%%%%%%%%%%%%%%%%%%%
	
		%
		%%%%%%%%%%%%%%%%%%%%%%%%%%%%%%%%%%%%%%%%%%%%%%%%%%%%%%%%%%%%%%%%%%%%%%
		%% USAR \cite{...} PARA INCLUIR REFERENCIAS BIBLIOGRAFICAS          %%
		%%  \cite{koon}  Para una sola                                      %%
		%%  \cite{oreg,khar,zvai,xjon,schn,pond} Para una lista             %%
		%%%%%%%%%%%%%%%%%%%%%%%%%%%%%%%%%%%%%%%%%%%%%%%%%%%%%%%%%%%%%%%%%%%%%%
		Thus we observe that this expected value is finite for all $v>0$ (also see \cite{koon,xjon,marg,schn,koha,issnic}).
		
		
		%%%%%%%%%%%%%%%%%%%%%%%%%%%%%%%%%%%%%%%%%%%%%%%%%%%%%%%%%%%%%%%%%%%%%%%%%%%%%%%%%%%%%%%%%%%
		%% FIGURAS                                                                               %%
		%% includegraphics: inserta la imagen                                                    %%
		%% caption: descripcion de la figura                                                     %%
		%% label: etiqueta para hacer referencia a la figura en el texto con la instrucción ref  %%
		%%%%%%%%%%%%%%%%%%%%%%%%%%%%%%%%%%%%%%%%%%%%%%%%%%%%%%%%%%%%%%%%%%%%%%%%%%%%%%%%%%%%%%%%%%%	
		
	
		%%%%%%%%%%%%%%%%%%%%%%%%%%%%%%%%%%%%%%%%%%%%%%%%%%%%%%%%%%%%%%%%%%%%%%%%%%%%%%%%%%%%%%%%%%
		%% TABLAS                                                                               %%
		%% caption: Descripción tabla                                                           %%
		%% \begin{tabular}{letras}: Indica numero de columnas.                                  %%
		%%    Una letra por columna, la letra indica la alineación de la columna:               %%
		%%    c center, l left, r right                                                         %%
		%% hline: Representa una linea como entre filas                                         %%
		%% \\: fin de fila                                                                      %%
		%% &: delimitador de celda                                                              %%
		%% label: etiqueta para hacer referencia a la tabla en el texto con la instrucción ref  %%
		%%%%%%%%%%%%%%%%%%%%%%%%%%%%%%%%%%%%%%%%%%%%%%%%%%%%%%%%%%%%%%%%%%%%%%%%%%%%%%%%%%%%%%%%%%
		
		
	
	%%%%%%%%%%%%%%%%%%%%%%%%%%%%%%%%%
	%% FIN DE EJEMPLO !!!!!!!!!!!! %%
	%%%%%%%%%%%%%%%%%%%%%%%%%%%%%%%%%
	
	%%%%%%%%%%%%%%%%%%%%%%%%%%%%%%%%%
	%% COMIENZO DEL DOCUMENTO REAL %%
	%%%%%%%%%%%%%%%%%%%%%%%%%%%%%%%%%
	
	\section{Introducción}

La aniridia es un trastorno congénito del desarrollo ocular caracterizado por la ausencia parcial o total del iris, la estructura que da color al ojo y que ayuda a regular la cantidad de luz que entra en la pupila. Esta condición no solo afecta la estética del ojo, sino que también puede conllevar una serie de problemas de visión, como fotofobia (sensibilidad a la luz), nistagmo (movimientos involuntarios del ojo) y una disminución de la agudeza visual. Los individuos con aniridia a menudo presentan malformaciones oculares asociadas, como cataratas, glaucoma y problemas en la córnea, lo que puede complicar aún más su salud visual y su calidad de vida \cite{Landsend2021}.

Además del compromiso ocular, la aniridia se asocia con múltiples patologías sistémicas, lo que sugiere que los mecanismos genéticos y funcionales detrás de esta enfermedad son complejos y multifactoriales. \cite{BLACK2022389} La investigación ha demostrado que la aniridia puede estar vinculada a síndromes como el síndrome de WAGR (Wilms tumor, Aniridia, Genitourinary abnormalities, and Range of developmental delays), que incluye no solo la aniridia, sino también anomalías renales y problemas de desarrollo \cite{lopezrelacion}. Otros síndromes asociados son el síndrome de Gillespie, que involucra aniridia y discapacidad mental, y el síndrome de Axenfeld-Rieger, que afecta a los ojos y otros órganos \cite{Law2011}. Estas asociaciones resaltan la importancia de investigar los mecanismos genéticos subyacentes a la aniridia y su relación con otras condiciones.

El gen principal asociado a la aniridia es el PAX6, que desempeña un papel crucial en el desarrollo y la diferenciación de los tejidos oculares. \cite{robles_lopez_2012} PAX6 es un factor de transcripción que regula la expresión de otros genes necesarios para la formación de diversas estructuras oculares. Las mutaciones en este gen han demostrado estar vinculadas no solo a la aniridia, sino también a otras anomalías oculares, como la displasia corneal y el glaucoma congénito \cite{CalvaoPires2014}. Sin embargo, la comprensión completa de cómo PAX6 interactúa con otros genes y qué otros genes están involucrados en la presentación clínica de la aniridia sigue siendo un área de estudio activa.

Además de PAX6, se han identificado otros genes relevantes en la patología de la aniridia, como FOXC1, WT1, COL4A1 y PITX2. Por ejemplo, las mutaciones en FOXC1 se han asociado con glaucoma y otras anomalías oculares en el contexto del síndrome de Axenfeld-Rieger \cite{Reis2023}. El gen WT1 es conocido por su papel en el desarrollo del riñón y en la formación de tumores, además de estar implicado en el desarrollo de anomalías oculares \cite{Pelletier1991}. COL4A1, que codifica una cadena del colágeno tipo IV, es esencial para la integridad de las membranas basales y se ha relacionado con varias anomalías oculares que afectan el desarrollo del segmento anterior \cite{Vahedi2011}. Finalmente, PITX2 es fundamental en la morfogénesis cardiovascular y en el desarrollo craneofacial, con mutaciones asociadas a trastornos que afectan la simetría y el desarrollo de órganos \cite{French2021}.

Este trabajo de investigación tiene como objetivo principal examinar las patologías asociadas al fenotipo de aniridia, centrándose en los genes que tienen una relación funcional con el gen PAX6. Mediante la identificación de genes relacionados, la creación de clusters \cite{ben1999clustering} y el análisis de sus mecanismos funcionales \cite{FloresCamacho2009}, buscamos profundizar en el entendimiento de cómo las alteraciones en estas interacciones génicas contribuyen al fenotipo de la aniridia y a otras enfermedades. Este enfoque no solo ampliará el conocimiento sobre la aniridia, sino que también proporcionará información valiosa sobre la interconexión de diversas condiciones patológicas y los mecanismos subyacentes que las vinculan.
	\section{Materiales y métodos}

En esta sección serán expuestos las diferentes herramientas y métodos utilizados para llevar a cabo el proyecto.

\subsection{\textbf{Materiales}}
\subsubsection{Human Phenotype Ontology}

La \textbf{Human Phenotype Ontology (HPO)} es unua base de datos estructurada que organiza y describe las características clínicas (fenotipos) observadas en humanos, utilizando términos estandarizados y jerárquicos \cite{Gargano2024}.

\subsubsection{STRING}

La \textbf{STRING Database} es una base de datos que integra información sobre interacciones entre proteínas, incluyendo interacciones físicas y funcionales, en una amplia variedad de organismos \cite{STRING2024}.

\subsubsection{Python}

\textbf{Python} es un lenguaje de programación versátil, fácil de aprender y ampliamente usado en desarrollo web, ciencia de datos, IA y más, gracias a su sintaxis clara y su gran ecosistema de herramientas \cite{Python2024}.


\subsection{\textbf{Metodología}}

\subsubsection{Obtención de genes, enfermedades y la red de interacción}

El primer paso que se llevó a cabo para comenzar el proyecto fue la obtención de los genes y enfermedades relacionadas con el fenotipo seleccionado. Para ello utilizamos la API de HPO, mediante una función solicitamos estos archivos introduciendo como parámetro el código de la aniridia (HP:000526). Si la respuesta es exitosa, guardamos tanto la lista de enfermedades como la lista de genes en nuestra carpeta de resultados. Más adelante utilizaremos el archivo relacionado con las patologías, ahora nos centraremos en la lista de genes.

Una vez obtenidos los genes, utilizamos la API de STRINGDB para descargar la red de interacción de proteínas. Con otra función hacemos un request a su servidor y solicitamos toda esta información, guardando el archivo red_inicial.tsv y la imagen de la red (red_inicial.png) en la carpeta Results. Cabe recalcar que nuestro organismo base es el Homo Sapiens y hemos establecido un valor de 0.4 como índice de confianza para las interacciones. Además, debemos limpiar el archivo .tsv para quedarnos solamente con las columnas de interés, es decir, aquellas que contienen los nombres de las proteínas que interaccionan. Guardamos la información filtrada en el archivo genes_filtrados.txt que emplearemos a continuación para analizar la red.

\subsubsection{Análisis de la red de interacción}
Todo lo anterior descrito fue programado en Python, ahora haremos uso de R y su librería iGraph. Importamos la librería y leemos la información de genes_filtrados.txt. Con varias funciones incluidas dentro de la librería, averiguamos distintos parámetros: si los nodos están completamente conectados, si el grafo está dirigido, el coeficiente de clustering local de los nodos, el coeficiente de clustering global, el grado de centralidad, la centralidad de intermediación, la modularidad, etc. Con toda esta información, elegimos cuáles son nuestros genes más interesantes para posteriormente propagar la red. Se tuvieron en cuenta concretamente tres variables: la centralidad de intermediación (por encima del percentil 0.8), el coeficiente de clustering local (menor al umbral de 0.5) y el grado de centralidad (superior al percentil 0.6). La elección de estos parámetros ha sido motivada por la necesidad de encontrar aquellos genes que puedan tener un papel importante en diferentes módulos biológicos, ya que una alta centralidad de intermediación indica la posibilidad de que esos genes actúen como puentes dentro de la red, además un coeficiente de clustering local bajo también señala que no están muy integrados en un solo módulo. Por otra parte, un alto grado de centralidad nos asegura que los genes seleccionados tengan cierta influencia dentro de la red, siendo hubs de la red. Así pues, tras escoger los nodos que cumplen con nuestros baremos, continuaremos con el siguiente paso, la propagación de red.

\subsubsection{Propagación de red}

Para analizar de manera exhaustiva los genes funcionalmente relevantes asociados con nuestro fenotipo de interés, aplicamos el algoritmo \textbf{DIAMOnD (Disease Module Detection)} a una red de interacción proteína-proteína (PPI). Este procedimiento nos permitió expandir un módulo génico inicial definido por el conjunto de genes semilla previamente seleccionado mediante la integración de información de interacción genética y funcional.
Los genes semilla iniciales, seleccionados por su relevancia biológica y su asociación conocida con el fenotipo en estudio, fueron: \textbf{WNT10B, SEM1, WT1, PAX6, y NF1}. Estos genes representan el núcleo inicial desde el cual se propagaron las interacciones para identificar genes adicionales funcionalmente relacionados.
Respecto a la red base, utilizamos un archivo que contiene todas las interacciones proteína-proteína humanas disponibles en la base de datos STRING.
El algoritmo DIAMOnD se utilizó para expandir el módulo génico inicial.  En este método, en cada iteración, el algoritmo evalúa los nodos vecinos de los genes ya presentes en el módulo y asigna un puntaje basado en el número de conexiones con los genes actuales del módulo. El nodo con el puntaje más alto se añade al módulo. Este proceso se repite hasta alcanzar el número máximo de nodos adicionales (en este caso, 200 nodos) o hasta que no queden candidatos relevantes. Una vez completada la expansión, se construyó un subgrafo que incluye todos los genes del módulo expandido junto con las aristas que los conectan.

\subsubsection{Clustering}


Para continuar con el análisis se desarrolló un algoritmo de clustering con el objetivo de identificar agrupaciones en la red de genes obtenida en el apartado anterior.


El \textbf{clustering} consiste en dividir un conjunto de datos en grupos o comunidades, de forma que los nodos de un mismo grupo tengan mayor relación entre sí que con los de los demás grupos.

Aplicado en el campo de la biología de sistemas, este método permite analizar redes complejas como las redes de interacción proteína-proteínas, rutas metabólicas y redes de regulación génica. A menudo, las comunidades identificadas con este tipo de clustering se corresponden a grupos de proteínas, genes (como es en este caso) o componentes moleculares que trabajan juntos para realizar una función específica.

Para este proyecto se emplea un código para realizar el clustering de los genes de la red usando la \textbf{biblioteca igraph}. En este lo primero que se hace es cargar los datos de las interacciones entre genes desde el archivo de la red obtenida previamente en formato tabular.

A continuación, se crea un grafo no dirigido con los datos previos. Se estudian algunas propiedades de la red.

Después, se calculan las \textbf{métricas de centralidad}, es decir, el \textbf{grado de centralidad} (mide el número de conexiones de cada nodo), la \textbf{centralidad de cercanía} (mide la distancia media entre un nodo y los demás) y la \textbf{conectividad} (mide la densidad de conexiones en el grafo).

Una vez obtenidas las medidas se aplican varios métodos de clustering. Estos son \textbf{Edge Betweenness Clustering}, basado en la intermediación de aristas e ir eliminando iterativamente las aristas con mayor betwenness; \textbf{Clustering Basado en el Algoritmo Fast Greedy}, este consiste en agrupar nodos que tienen más conexiones entre ellos que con otros, optimizando de manera eficiente la modularidad; y \textbf{Clustering Basado en el Algoritmo Louvain}, el cual encuentra comunidades en redes grandes optimizando su estructura jerárquica para maximizar la modularidad.

Además, se realiza la identificación de comunidades con \textbf{Link Communities}, método que considera las relaciones entre aristas, es decir, entre nodos.

Estos resultados se guardan como imágenes en la carpeta del proyecto "results" y se añadirán en el apartado de resultados de este informe.

Por último, se buscan vecinos interesantes de los genes de interés en el estudio \textbf{(WNT10B, WT1, SEM1, PAX6, NF1)}. Estos se guardan en un archivo plano que se empleará para el enriquecimiento funcional.

\subsubsection{Enriquecimiento funcional de los clusters}


Para profundizar en la caracterización funcional del cluster seleccionado en el apartado anterior, hemos llevado a cabo un análisis de enriquecimiento funcional utilizando un enfoque basado en la base de datos STRINGdb. En este análisis, evaluamos específicamente dos categorías: procesos biológicos (Process) y ontología del fenotipo humano (HPO, Human Phenotype Ontology). Además de poder evaluar los procesos biológicos en los que están envueltos estos genes, otro de los objetivos era determinar si entre los fenotipos resultantes del análisis de enriquecimiento se encuentran aquellos descritos en HPO como relacionados con el fenotipo de interés.

Durante el proceso se ha llevado a cabo la conversión a identificadores UniProt de los genes del cluster mediante el uso de la API de MyGene.info, asegurando compatibilidad con la API de STRINGdb.

Posteriormente, dichos identificadores UniProt fueron sometidos al análisis de enriquecimiento utilizando la base de datos STRINGdb, con un enfoque en las categorías Process y HPO. 

En este proceso nosotros consideramos oportuno establecer un umbral de significancia estadística de \textbf{ FDR  \texttt< 0.001}, para identificar términos significativamente enriquecidos.
Se evaluaron las asociaciones entre los genes del cluster y términos fenotípicos relevantes descritos en HPO.

Como resultado de la ejecución del algoritmo desarrollado se obtienen dos archivos en formato csv. Uno recoge los resultados para la categoría HPO y los de la categoría Process.

	\section{Resultados}

\subsection{Obtención de genes, enfermedades y la red de interacción}
Al hacer las solicitudes a HPO obtenemos 45 genes y 29 enfermedades asociadas a \textbf{Aniridia}. Después obtenemos la red de interacción que visualizamos aquí abajo.

\begin{figure}[h] % [h] indica que queremos la imagen aquí, en la posición actual
	\centering
	\includegraphics[width=1\textwidth]{figures/red_interaccion_aniridia.png} % Especifica la ruta y el tamaño
	\caption{Red de interacción con los genes asociados al fenotipo HP:0000526} % Agrega una leyenda si deseas
	\label{fig:m}
\end{figure}

\subsection{Análisis de la red inicial}
En R y con la librería iGraph hemos estudiado nuestra red y nos indica que todos los nodos del conjunto están conectados y no se trata de un grafo dirigido. Además, calculamos que el coeficiente de clustering global es 0.68, la conectividad de 0.16, la modularidad 0.42, la dispersión de la red es de 0.84, la longitud promedio de camino es 2.74, el diámetro de la red es igual a 6 y el coeficiente de asortividad es de 0.49. En la tabla \ref{tab:parametros_red} podemos ver todos los datos recogidos.

\begin{table}[ht]
	\centering
	\begin{tabular}{|c|c|}
		\hline
		\textbf{Parámetro} & \textbf{Valor} \\ \hline
		Conexión de nodos & Todos nodos conectados \\ \hline
		Tipo de grafo & No dirigido \\ \hline
		Coeficiente de clustering global & 0.68 \\ \hline
		Conectividad & 0.16 \\ \hline
		Modularidad & 0.42 \\ \hline
		Dispersión de la red & 0.84 \\ \hline
		Longitud promedio de camino & 2.74 \\ \hline
		Diámetro de la red & 6 \\ \hline
		Coeficiente de asortividad & 0.49 \\ \hline
	\end{tabular}
	\caption{Parámetros estructurales de la red de interacción}
	\label{tab:parametros_red}
\end{table}

Además, para cada nodo hemos calculado el coeficiente de clustering local, el grado de centralidad y su centralidad de intermediación, toda esta información puede observarse en la tabla \ref{tab:parametros_red}. 
\begin{table}[ht]
	\centering
	\begin{tabular}{|c|c|c|c|}
		\hline
		\textbf{Nombre} & \textbf{Índice Intermediación} & \textbf{Coef. Clustering} & \textbf{Grado Centralidad}\\ \hline
		TRIP13 & 0 & NA & 1 \\
		SALL4 & 0.2372 & 0 & 4 \\
		SLC25A11 & 0.0005 & 0.9722 & 9 \\
		TRIM28 & 0.0197 & 0 & 2 \\
		VHL & 0.0826 & 0.3333 & 3 \\
		TMEM127 & 0.0137 & 0.8939 & 12 \\
		KIF1B & 0.0080 & 0.9556 & 10 \\
		TP63 & 0 & 1 & 3 \\
		SDHA & 0.1029 & 0.7143 & 14 \\
		TRIM44 & 0.1025 & 0 & 2 \\
		LDHD & 0 & 1 & 2 \\
		WNT10B & 0.2597 & 0.3333 & 6 \\
		SDHAF2 & 0.0137 & 0.8939 & 12 \\
		REST & 0.1345 & 0 & 4 \\
		MDH2 & 0.0294 & 0.8205 & 13 \\
		FOXE3 & 0 & 1 & 2 \\
		DLST & 0 & 1 & 6 \\
		RET & 0.2009 & 0.7121 & 12 \\
		NF1 & 1 & 0.4762 & 15 \\
		MAX & 0.0041 & 0.9556 & 10 \\
		FH & 0.0362 & 0.7455 & 11 \\
		SDHC & 0.3483 & 0.7143 & 14 \\
		BTRC & 0.1152 & 0.3333 & 3 \\
		MAFB & 0.0914 & 0 & 2 \\
		SDHB & 0.1759 & 0.6286 & 15 \\
		WT1 & 0.5058 & 0.2857 & 7 \\
		BRCA2 & 0.1960 & 0.4 & 5 \\
		GPC3 & 0.2936 & 0.1667 & 4 \\
		ELP4 & 0 & 1 & 2 \\
		DLX6 & 0.0862 & 0.7 & 5 \\
		SEM1 & 0.6555 & 0.2381 & 7 \\
		PAX6 & 0.8730 & 0.1389 & 9 \\
		FOXC1 & 0 & 1 & 2 \\
		DLX5 & 0.1873 & 0.4667 & 6 \\
		PITX2 & 0.1521 & 0.3333 & 4 \\
		CHN1 & 0 & NA & 1 \\
		SDHD & 0.0294 & 0.8205 & 13 \\
		FBXW4 & 0.0762 & 0.6 & 5 \\
		PORCN & 0 & NA & 1 \\
		POU6F2 & 0.0755 & 0.3333 & 3 \\
		CTCFL & 0 & NA & 1 \\ \hline
	\end{tabular}
	\caption{Parámetros de los nodos de la red}
	\label{tab:parametros_nodos_red}
\end{table}

Ahora vamos a filtrar con un percentil de 0.6 en degree, un percentil de 0.8 en betweenness y un clustering mayor a 0.5 y obtenemos 5 genes que cumplen estas condiciones: WNT10B, NF1, WT1, SEM1, PAX6.

\subsection{Propagación de la red}

A partir de los genes semilla WNT10B, WT1, SEM1, PAX6, y NF1, se generó una red de interacciones proteína-proteína (PPI) utilizando datos de STRINGdb. Se estableció un umbral de puntaje de interacción de 700 para incluir únicamente conexiones de alta confianza. La red inicial centrada en los genes semilla fue posteriormente expandida mediante el algoritmo DIAMOnD, incluyendo 200 nodos adicionales que se encuentran altamente conectados con los genes semilla y entre sí.

La red generada señala importantes interacciones entre los genes semilla y otros desconocidos (como la relación WNT10B - TGFB2), lo cual permite continuar el estudio hacia el objetivo del artículo.


\subsection{Clustering}

\begin{figure}[!h] % [h] indica que queremos la imagen aquí, en la posición actual
	\centering
	\includegraphics[width=1\textwidth]{figures/toda_figuras_clustering.png}
	\caption{Figuras obtenidas al aplicar clustering con distintos métodos}
	\label{clustering}
\end{figure}

Para esto usaremos la figura \ref{clustering}. Principalmente el estudio de los resultados se centrará en los genes semilla (WNT10B, WT1, SEM1, PAX6, NF1). En común, se puede observar que SEM1 aparece en varisa posiciones destacadas en los grafos; WT1 y PAX6 están en regiones densas (centrales) de las comunidades, lo que infica que tienen muchas conexiones internas; y NF1 se sitúa en los bordes de las comunidades, por lo que podría actuar como puente entre diferentes comunidades; y WNT10B también es central en algunas comunidades.


\subsection{Enriquecimiento y combinación de fenotipos}

Al realizar un análisis de enriquecimiento considerando la categoría “Process” generamos un archivo que nos permite comprender en qué procesos biológicos clave están involucrados los genes del cluster seleccionado. Esto ayuda a identificar rutas metabólicas, mecanismos celulares y funciones moleculares que pueden estar asociados con el fenotipo de interés.

Por otra parte, realizar el enriquecimiento funcional considerando categoría “HPO” es crucial para identificar cómo los genes analizados están asociados con características fenotípicas específicas, especialmente aquellas relacionadas con enfermedades humanas. 
Por último, combinar los fenotipos resultantes del enriquecimiento y los asociados a las enfermedades asociadas a nuestro fenotipo permite confirmar si los términos fenotípicos enriquecidos en los clusters tienen una relación directa con las enfermedades previamente identificadas. Si hay coincidencias, esto valida que los genes analizados están realmente relacionados con el fenotipo de interés.

Además, el establecer identificaciones entre las relaciones genotipo-fenotipo ayuda a establecer conexiones concretas entre los genes y las manifestaciones clínicas observadas en pacientes. Esto es especialmente útil para comprender cómo los genes influyen en el desarrollo de ciertas enfermedades o características clínicas.

Finalmente, podemos ver que 284 fenotipos de los 668 obtenidos tras el enriquecimiento están relacionados con enfermedades relacionadas con nuestro fenotipo a estudiar. Este resultado sugiere que el análisis de enriquecimiento funcional está capturando información biológica y clínica relevante. La conexión entre los fenotipos obtenidos y las enfermedades relacionadas con la aniridia valida que los genes analizados están efectivamente implicados en procesos relacionados con el fenotipo de interés. Este hallazgo podría abrir la puerta a la identificación de procesos centrales o vías moleculares críticas en estas enfermedades.
Por otra parte,  los fenotipos compartidos pueden ayudar a priorizar genes o grupos de genes específicos para estudios adicionales. Estos genes podrían ser considerados como potenciales biomarcadores o dianas terapéuticas para tratar tanto la aniridia como las enfermedades relacionadas.
Los fenotipos no relacionados también son interesantes, ya que podrían reflejar aspectos únicos del fenotipo de interés o nuevas áreas de estudio.






	\section{Discusión}

El análisis de la \textbf{aniridia} como fenotipo central permitió identificar múltiples asociaciones relevantes con enfermedades genéticas descritas en bases de datos especializadas como OMIM y Orphanet. En los datos analizados, se observa que la aniridia (HPO:0000526) está presente en una amplia variedad de condiciones, reflejando su relevancia clínica y biológica. Estas incluyen, entre otras, *Axenfeld-Rieger syndrome, type 1* (OMIM:180500), *D-lactic aciduria with susceptibility to gout* (OMIM:245450), *Nephroblastoma* (ORPHA:654) y *Chromosome 11p13 deletion syndrome, distal* (OMIM:616902). Cada una de estas enfermedades aporta perspectivas complementarias sobre los mecanismos genéticos y moleculares asociados a la aniridia.

El gen \textit{PAX6}, ampliamente reconocido como un regulador maestro en el desarrollo ocular, surge como un elemento central en el análisis de redes de interacción genética. Este gen no solo está directamente relacionado con la aniridia, sino que también actúa como un nodo crítico en las redes moleculares que controlan procesos como la diferenciación de tejidos oculares, el desarrollo neuronal y la señalización celular. Adicionalmente, genes como \textit{WT1} y \textit{NF1}, identificados en las redes extendidas generadas, refuerzan la idea de que la aniridia no es un fenotipo aislado, sino que forma parte de una red compleja de interacciones genéticas con implicaciones multisistémicas.

La diversidad de enfermedades asociadas a la aniridia destaca la importancia de abordar este fenotipo desde una perspectiva interdisciplinaria. Por ejemplo, en condiciones como el *Nephroblastoma* y el *Chromosome 11p13 deletion syndrome*, la aniridia es acompañada de fenotipos adicionales como hipertensión, hemihipertrofia y retraso en el desarrollo psicomotor. Estas asociaciones subrayan la necesidad de considerar no solo los fenotipos primarios, sino también los secundarios o relacionados, para realizar diagnósticos más precisos.

El análisis funcional de los fenotipos asociados también resalta que un número significativo de enfermedades relacionadas con la aniridia involucran procesos moleculares como la regulación del desarrollo embrionario, la señalización del factor de crecimiento transformante beta (TGF-β), y la reparación del ADN. Esto sugiere que los mecanismos subyacentes a la aniridia podrían incluir disfunciones en rutas moleculares comunes que contribuyen tanto a defectos estructurales como a susceptibilidades a enfermedades malignas.

Aunque el presente análisis se centra en asociaciones conocidas y documentadas en bases de datos como HPO, OMIM y Orphanet, es fundamental reconocer las limitaciones de estas fuentes. Las interacciones gen-fenotipo podrían estar infraestimadas debido a la falta de datos en poblaciones específicas o por la subrepresentación de variantes genéticas raras en estas bases de datos. Por lo tanto, futuros estudios deberían incluir enfoques complementarios, como el análisis transcriptómico y proteómico en tejidos afectados, para identificar patrones moleculares específicos de la aniridia.

En conclusión, la \textbf{aniridia} no solo es un marcador clínico de enfermedades genéticas específicas, sino también una ventana a la comprensión de redes moleculares más amplias que subyacen a defectos del desarrollo y condiciones multisistémicas. La integración de herramientas bioinformáticas, redes moleculares y datos clínicos puede mejorar significativamente el diagnóstico y manejo de estas enfermedades, abriendo además oportunidades para desarrollar terapias dirigidas basadas en los mecanismos subyacentes identificados.

	\section{Conclusiones}

El enfoque utilizado en este estudio demuestra el potencial de las herramientas bioinformáticas para desentrañar la complejidad de fenotipos clínicos como la aniridia. A través de una combinación de análisis de red, clustering y enriquecimiento funcional, hemos identificado genes clave y procesos moleculares relacionados con los fenotipos de la enfermedad. Este trabajo no solo confirma la validez del modelo aplicado, sino que también destaca nuevas áreas de investigación y posibles aplicaciones clínicas, contribuyendo al conocimiento de la base genética y molecular de la aniridia.
	
	
	%%%%%%%%%%%%%%%%%%%%%%%%%%%%%%%%%%%%%%%%%%%%%%
	%% OTRA INFORMACIÓN                         %%
	%%%%%%%%%%%%%%%%%%%%%%%%%%%%%%%%%%%%%%%%%%%%%%
	
	\begin{backmatter}
	
		\section*{Abreviaciones}%% if any
			Indicar lista de abreviaciones mostrando cada acrónimo a que corresponde
		
		\section*{Disponibilidad de datos y materiales}%% if any
			Debéis indicar aquí un enlace a vuestro repositorio de github.
		
		\section*{Contribución de los autores}
			Usando las iniciales que habéis definido al comienzo del documento, debeis indicar la contribución al proyecto en el estilo:
			J.E : Encargado del análisis de coexpresión con R, escritura de resultados; J.R.S : modelado de red con python y automatizado del código, escritura de métodos; ...
			OJO: que sea realista con los registros que hay en vuestros repositorios de github. 
		
		
		%%%%%%%%%%%%%%%%%%%%%%%%%%%%%%%%%%%%%%%%%%%%%%%%%%%%%%%%%%%%%%%%%%%%%%%%%%%%%%%%%%%%%%%%
		%% BIBLIOGRAFIA: no teneis que tocar nada, solo sustituir el archivo bibliography.bib %%
		%% por el que hayais generado vosotros                                                %%
		%%%%%%%%%%%%%%%%%%%%%%%%%%%%%%%%%%%%%%%%%%%%%%%%%%%%%%%%%%%%%%%%%%%%%%%%%%%%%%%%%%%%%%%%
		
		\bibliographystyle{bmc-mathphys} % Style BST file (bmc-mathphys, vancouver, spbasic).
		\bibliography{bibliography}      % Bibliography file (usually '*.bib' )
	
	\end{backmatter}
\end{document}
