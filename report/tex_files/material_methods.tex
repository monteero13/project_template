\section{Materiales y métodos}

En esta sección serán expuestos las diferentes herramientas y métodos utilizados para llevar a cabo el proyecto.

\subsection{\textbf{Materiales}}
\subsubsection{Human Phenotype Ontology}

La \textbf{Human Phenotype Ontology (HPO)} es unua base de datos estructurada que organiza y describe las características clínicas (fenotipos) observadas en humanos, utilizando términos estandarizados y jerárquicos \cite{Gargano2024}.

\subsubsection{STRING}

La \textbf{STRING Database} es una base de datos que integra información sobre interacciones entre proteínas, incluyendo interacciones físicas y funcionales, en una amplia variedad de organismos \cite{STRING2024}.

\subsubsection{Python}

\textbf{Python} es un lenguaje de programación versátil, fácil de aprender y ampliamente usado en desarrollo web, ciencia de datos, IA y más, gracias a su sintaxis clara y su gran ecosistema de herramientas \cite{Python2024}.


\subsection{\textbf{Metodología}}

\subsubsection{Obtención de genes}

El primer paso que se llevó a cabo para comenzar el proyecto es la obtención de los genes relacionados con el fenotipo seleccionado. Para ello se visitó directamente la web de HPO (https://hpo.jax.org/) y se introdujo el identificador del fenotipo, HP:0000526 en este caso. Como resultado de la busqueda se obtuvo el conjunto de genes relacionados con la aniridia. Finalmente descargamos el conjunto de genes resultante de HPO.  


\subsubsection{Obtención de la red de interacciones inicial}

Tras descargar los genes, se utilizó la pagina web de STRING para obtener su red de interacciones (https://string-db.org/). Para ello es necesario seleccionar la opción para introducir múltiples proteínas y especificar la especie, en este caso Homo Sapiens. Se introdujo la lista de genes obtenida de HPO. Finalmente, dandole a exportar aparecen varias opciones. En el caso de este estudio se descargó la imagenen formato PNG de la red de interacciones. Además se descargaron los datos de la red como salida de texto en formato tsv.


\subsubsection{Análisis de la red de interacciones}

En este punto del proyecto se llevó a cabo un análisis exaustivo de la red de interacciones de todos los genes implicados en el fenotipo. El análisis de dicha red  implica examinar cómo los genes están conectados entre sí para identificar patrones, relaciones funcionales y estructuras que revelen información clave. 

Se calcularon métricas clave como el grado de los nodos, la centralidad y la modularidad, con el objetivo de identificar los nodos más importantes dentro de la red. A través del análisis estructural y funcional, se detectaron los nodos clave, aquellos que desempeñan un papel central en la red.

El resto del proyecto gira en torno a los genes clave seleccionados tras el análisis de la red.

\subsubsection{Propagación de red}

Con los genes clave seleccionados creamos una red de propagación. Para llevar a cabo esta tarea fue necesario desarrollar un algoritmo que permitiese propagar una red a partir de unos genes semilla.


\subsubsection{Clustering}


Para continuar con el análisis se desarrolló un algoritmo de clustering con el objetivo de identificar agrupaciones en la red de genes obtenida en el apartado anterior.


El \textbf{clustering} consiste en dividir un conjunto de datos en grupos o comunidades, de forma que los nodos de un mismo grupo tengan mayor relación entre sí que con los de los demás grupos.

Aplicado en el campo de la biología de sistemas, este método permite analizar redes complejas como las redes de interacción proteína-proteínas, rutas metabólicas y redes de regulación génica. A menudo, las comunidades identificadas con este tipo de clustering se corresponden a grupos de proteínas, genes (como es en este caso) o componentes moleculares que trabajan juntos para realizar una función específica.

Para este proyecto se emplea un código para realizar el clustering de los genes de la red usando la \textbf{biblioteca igraph}. En este lo primero que se hace es cargar los datos de las interacciones entre genes desde el archivo de la red obtenida previamente en formato tabular.

A continuación, se crea un grafo no dirigido con los datos previos. Se estudian algunas propiedades de la red.

Después, se calculan las \textbf{métricas de centralidad}, es decir, el \textbf{grado de centralidad} (mide el número de conexiones de cada nodo), la \textbf{centralidad de cercanía} (mide la distancia media entre un nodo y los demás) y la \textbf{conectividad} (mide la densidad de conexiones en el grafo).

Una vez obtenidas las medidas se aplican varios métodos de clustering. Estos son \textbf{Edge Betweenness Clustering}, basado en la intermediación de aristas e ir eliminando iterativamente las aristas con mayor betwenness; \textbf{Clustering Basado en el Algoritmo Fast Greedy}, este consiste en agrupar nodos que tienen más conexiones entre ellos que con otros, optimizando de manera eficiente la modularidad; y \textbf{Clustering Basado en el Algoritmo Louvain}, el cual encuentra comunidades en redes grandes optimizando su estructura jerárquica para maximizar la modularidad.

Además, se realiza la identificación de comunidades con \textbf{Link Communities}, método que considera las relaciones entre aristas, es decir, entre nodos.

Estos resultados se guardan como imágenes en la carpeta del proyecto "results" y se añadirán en el apartado de resultados de este informe.

Por último, se buscan vecinos interesantes de los genes de interés en el estudio \textbf{(WNT10B, WT1, SEM1, PAX6, NF1)}. Estos se guardan en un archivo plano que se empleará para el enriquecimiento funcional.

\subsubsection{Enriquecimiento funcional de los clusters}