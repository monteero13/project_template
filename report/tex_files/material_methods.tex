\section{Materiales y métodos}

En esta sección serán expuestos las diferentes herramientas y métodos utilizados para llevar a cabo el proyecto.

\subsection{\textbf{Materiales}}
\subsubsection{Human Phenotype Ontology}

La \textbf{Human Phenotype Ontology (HPO)} es unua base de datos estructurada que organiza y describe las características clínicas (fenotipos) observadas en humanos, utilizando términos estandarizados y jerárquicos \cite{Gargano2024}.

\subsubsection{STRING}

La \textbf{STRING Database} es una base de datos que integra información sobre interacciones entre proteínas, incluyendo interacciones físicas y funcionales, en una amplia variedad de organismos \cite{STRING2024}.

\subsubsection{Python}

\textbf{Python} es un lenguaje de programación versátil, fácil de aprender y ampliamente usado en desarrollo web, ciencia de datos, IA y más, gracias a su sintaxis clara y su gran ecosistema de herramientas \cite{Python2024}.


\subsection{\textbf{Metodología}}

\subsubsection{Obtención de genes, enfermedades y la red de interacción}

El primer paso que se llevó a cabo para comenzar el proyecto fue la obtención de los genes y enfermedades relacionadas con el fenotipo seleccionado. Para ello utilizamos la API de HPO, mediante una función solicitamos estos archivos introduciendo como parámetro el código de la aniridia (HP:000526). Si la respuesta es exitosa, guardamos tanto la lista de enfermedades como la lista de genes en nuestra carpeta de resultados. Más adelante utilizaremos el archivo relacionado con las patologías, ahora nos centraremos en la lista de genes.

Una vez obtenidos los genes, utilizamos la API de STRINGDB para descargar la red de interacción de proteínas. Con otra función hacemos un request a su servidor y solicitamos toda esta información, guardando el archivo red_inicial.tsv y la imagen de la red (red_inicial.png) en la carpeta Results. Cabe recalcar que nuestro organismo base es el Homo Sapiens y hemos establecido un valor de 0.4 como índice de confianza para las interacciones. Además, debemos limpiar el archivo .tsv para quedarnos solamente con las columnas de interés, es decir, aquellas que contienen los nombres de las proteínas que interaccionan. Guardamos la información filtrada en el archivo genes_filtrados.txt que emplearemos a continuación para analizar la red.

\subsubsection{Análisis de la red de interacción}
Todo lo anterior descrito fue programado en Python, ahora haremos uso de R y su librería iGraph. Importamos la librería y leemos la información de genes_filtrados.txt. Con varias funciones incluidas dentro de la librería, averiguamos distintos parámetros: si los nodos están completamente conectados, si el grafo está dirigido, el coeficiente de clustering local de los nodos, el coeficiente de clustering global, el grado de centralidad, la centralidad de intermediación, la modularidad, etc. Con toda esta información, elegimos cuáles son nuestros genes más interesantes para posteriormente propagar la red. Se tuvieron en cuenta concretamente tres variables: la centralidad de intermediación (por encima del percentil 0.8), el coeficiente de clustering local (menor al umbral de 0.5) y el grado de centralidad (superior al percentil 0.6). La elección de estos parámetros ha sido motivada por la necesidad de encontrar aquellos genes que puedan tener un papel importante en diferentes módulos biológicos, ya que una alta centralidad de intermediación indica la posibilidad de que esos genes actúen como puentes dentro de la red, además un coeficiente de clustering local bajo también señala que no están muy integrados en un solo módulo. Por otra parte, un alto grado de centralidad nos asegura que los genes seleccionados tengan cierta influencia dentro de la red, siendo hubs de la red. Así pues, tras escoger los nodos que cumplen con nuestros baremos, continuaremos con el siguiente paso, la propagación de red.

\subsubsection{Propagación de red}

Para analizar de manera exhaustiva los genes funcionalmente relevantes asociados con nuestro fenotipo de interés, aplicamos el algoritmo \textbf{DIAMOnD (Disease Module Detection)} a una red de interacción proteína-proteína (PPI). Este procedimiento nos permitió expandir un módulo génico inicial definido por el conjunto de genes semilla previamente seleccionado mediante la integración de información de interacción genética y funcional.
Los genes semilla iniciales, seleccionados por su relevancia biológica y su asociación conocida con el fenotipo en estudio, fueron: \textbf{WNT10B, SEM1, WT1, PAX6, y NF1}. Estos genes representan el núcleo inicial desde el cual se propagaron las interacciones para identificar genes adicionales funcionalmente relacionados.
Respecto a la red base, utilizamos un archivo que contiene todas las interacciones proteína-proteína humanas disponibles en la base de datos STRING.
El algoritmo DIAMOnD se utilizó para expandir el módulo génico inicial.  En este método, en cada iteración, el algoritmo evalúa los nodos vecinos de los genes ya presentes en el módulo y asigna un puntaje basado en el número de conexiones con los genes actuales del módulo. El nodo con el puntaje más alto se añade al módulo. Este proceso se repite hasta alcanzar el número máximo de nodos adicionales (en este caso, 200 nodos) o hasta que no queden candidatos relevantes. Una vez completada la expansión, se construyó un subgrafo que incluye todos los genes del módulo expandido junto con las aristas que los conectan.

\subsubsection{Clustering}


Para continuar con el análisis se desarrolló un algoritmo de clustering con el objetivo de identificar agrupaciones en la red de genes obtenida en el apartado anterior.


El \textbf{clustering} consiste en dividir un conjunto de datos en grupos o comunidades, de forma que los nodos de un mismo grupo tengan mayor relación entre sí que con los de los demás grupos.

Aplicado en el campo de la biología de sistemas, este método permite analizar redes complejas como las redes de interacción proteína-proteínas, rutas metabólicas y redes de regulación génica. A menudo, las comunidades identificadas con este tipo de clustering se corresponden a grupos de proteínas, genes (como es en este caso) o componentes moleculares que trabajan juntos para realizar una función específica.

Para este proyecto se emplea un código para realizar el clustering de los genes de la red usando la \textbf{biblioteca igraph}. En este lo primero que se hace es cargar los datos de las interacciones entre genes desde el archivo de la red obtenida previamente en formato tabular.

A continuación, se crea un grafo no dirigido con los datos previos. Se estudian algunas propiedades de la red.

Después, se calculan las \textbf{métricas de centralidad}, es decir, el \textbf{grado de centralidad} (mide el número de conexiones de cada nodo), la \textbf{centralidad de cercanía} (mide la distancia media entre un nodo y los demás) y la \textbf{conectividad} (mide la densidad de conexiones en el grafo).

Una vez obtenidas las medidas se aplican varios métodos de clustering. Estos son \textbf{Edge Betweenness Clustering}, basado en la intermediación de aristas e ir eliminando iterativamente las aristas con mayor betwenness; \textbf{Clustering Basado en el Algoritmo Fast Greedy}, este consiste en agrupar nodos que tienen más conexiones entre ellos que con otros, optimizando de manera eficiente la modularidad; y \textbf{Clustering Basado en el Algoritmo Louvain}, el cual encuentra comunidades en redes grandes optimizando su estructura jerárquica para maximizar la modularidad.

Además, se realiza la identificación de comunidades con \textbf{Link Communities}, método que considera las relaciones entre aristas, es decir, entre nodos.

Estos resultados se guardan como imágenes en la carpeta del proyecto "results" y se añadirán en el apartado de resultados de este informe.

Por último, se buscan vecinos interesantes de los genes de interés en el estudio \textbf{(WNT10B, WT1, SEM1, PAX6, NF1)}. Estos se guardan en un archivo plano que se empleará para el enriquecimiento funcional.

\subsubsection{Enriquecimiento funcional de los clusters}


Para profundizar en la caracterización funcional del cluster seleccionado en el apartado anterior, hemos llevado a cabo un análisis de enriquecimiento funcional utilizando un enfoque basado en la base de datos STRINGdb. En este análisis, evaluamos específicamente dos categorías: procesos biológicos (Process) y ontología del fenotipo humano (HPO, Human Phenotype Ontology). Además de poder evaluar los procesos biológicos en los que están envueltos estos genes, otro de los objetivos era determinar si entre los fenotipos resultantes del análisis de enriquecimiento se encuentran aquellos descritos en HPO como relacionados con el fenotipo de interés.

Durante el proceso se ha llevado a cabo la conversión a identificadores UniProt de los genes del cluster mediante el uso de la API de MyGene.info, asegurando compatibilidad con la API de STRINGdb.

Posteriormente, dichos identificadores UniProt fueron sometidos al análisis de enriquecimiento utilizando la base de datos STRINGdb, con un enfoque en las categorías Process y HPO. 

En este proceso nosotros consideramos oportuno establecer un umbral de significancia estadística de \textbf{ FDR  \texttt< 0.001}, para identificar términos significativamente enriquecidos.
Se evaluaron las asociaciones entre los genes del cluster y términos fenotípicos relevantes descritos en HPO.

Como resultado de la ejecución del algoritmo desarrollado se obtienen dos archivos en formato csv. Uno recoge los resultados para la categoría HPO y los de la categoría Process.



\subsubsection{Obtencion de fenotipos comunes}

Para finalizar el proyecto es de interés ver si los fenotipos obtenidos en el enriquecimiento anterior están relacionados con las enfermedades relacionadas con nuestro fenotipo obtenidas al comienzo del proyecto. Para ello desarrollamos un script donde se obtuvieron todos los fenotipos asociados a dichas enfermedades. Posteriormente utilizamos otro script para ver los fenotipos comunes entre los obtenidos en el proceso de enriquecimiento y los relacionados con las enfermedades asociadas a la aniridia. Con esto buscamos indagar entre la relación entre estos fenotipos comunas, las enfermedades seleccionadas y nuestro fenotipo.
