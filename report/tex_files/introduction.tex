\section{Introducción}

La aniridia es una enfermedad ocular rara, de origen genético, caracterizada por la ausencia parcial o total del iris, lo que genera una serie de complicaciones visuales severas. Este trastorno no solo afecta la estructura ocular, sino que también se asocia a una serie de anomalías sistémicas y síndromes relacionados, como el síndrome de WAGR (Wilms tumor, Aniridia, Genitourinary anomalies, and Retardation), glaucoma, cataratas y disfunciones en el desarrollo del sistema nervioso central. La base genética de la aniridia está principalmente vinculada a mutaciones en el gen PAX6, un gen maestro en el desarrollo ocular y cerebral. Sin embargo, se ha identificado que otros genes también pueden estar implicados, lo que sugiere una complejidad genética que trasciende el mero defecto en un solo locus.

El estudio de las enfermedades relacionadas con la aniridia es crucial, ya que el diagnóstico temprano y la comprensión de los mecanismos moleculares que subyacen a estas patologías pueden tener un impacto significativo en la calidad de vida de los pacientes. La pérdida de visión progresiva, las dificultades para la movilidad y el acceso limitado a tratamientos personalizados son solo algunas de las dificultades que enfrentan los pacientes y sus familias. Además, la incidencia de neoplasias, como el tumor de Wilms, en individuos con aniridia congénita subraya la necesidad de estudios que no solo aborden los aspectos oftalmológicos, sino también las manifestaciones sistémicas de este trastorno.

En la sociedad actual, los pacientes con aniridia y enfermedades relacionadas enfrentan barreras significativas en términos de acceso a cuidados especializados, seguimiento clínico adecuado y opciones terapéuticas que aborden la totalidad del fenotipo. Este panorama es aún más preocupante en áreas con sistemas de salud menos desarrollados, donde el diagnóstico genético y las intervenciones tempranas suelen ser limitados. La falta de sensibilización y el desconocimiento de las implicaciones multisistémicas de la aniridia exacerban estas dificultades, contribuyendo a un problema de salud pública que a menudo no recibe la atención adecuada.

Por lo tanto, este trabajo tiene como objetivo explorar en detalle las relaciones entre los diferentes genes involucrados en la aniridia y las enfermedades relacionadas, analizando cómo las mutaciones en estos genes pueden provocar no solo disfunciones oculares, sino también otros trastornos del desarrollo. En particular, se plantea la pregunta: ¿cómo contribuyen las variaciones genéticas más allá de PAX6 a la manifestación del fenotipo complejo de la aniridia? Además, se discutirá el impacto social de estas enfermedades y se propondrán posibles estrategias para mejorar tanto el diagnóstico como el tratamiento de los pacientes afectados. Este análisis multidisciplinar busca ofrecer una visión integral que sirva como punto de partida para futuras investigaciones y para la implementación de políticas de salud pública más inclusivas y efectivas.