\section{Introducción}

Introducción

La aniridia es una condición que se define como la parcial o completa hipoplasia del iris, el anillo coloreado que se sitúa alrededor de la pupila dentro del ojo. Este fenotipo puede ser ocasionado por un traumatismo, una cirugía complicada o un defecto genético. 
Este fenotipo se presenta como resultado de una mutación que impide el correcto desarrollo del globo ocular durante las primeras semanas de gestación. La aniridia se presenta como una enfermedad congénita aislada o asociada a otras patologías como el síndrome de WAGR, el tumor de Wilms, el síndrome de Gillespie o el síndrome de Axenfeld-Rieger.

Información sobre los genes a estudiar

La aniridia tiene una estrecha relación con el gen PAX-6, aunque no es el único gen asociado. A continuación comentaremos algunos de estos genes: 
Gen PAX-6: este es un gen esencial debido a que codifica una proteína que actúa como un factor de transcripción, regulando la expresión de otros genes importantes en el desarrollo de diversas estructuras oculares como el iris.

