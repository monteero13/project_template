\section{Introducción}

La aniridia es un trastorno congénito del desarrollo ocular caracterizado por la
ausencia parcial o total del iris, la estructura que da color al ojo y que regula la
cantidad de luz que entra en la pupila. Esta condición, además de afectar la estética
del ojo, puede causar una serie de problemas visuales significativos, como fotofobia (sensibilidad a la luz), nistagmo (movimientos involuntarios de los ojos) y
disminución de la agudeza visual. Los individuos con aniridia a menudo presentan
malformaciones oculares asociadas, como cataratas, glaucoma (enfermedad ocular
que daña el nervio óptico, esencial para la visión) y problemas en la córnea, lo que agrava su
salud visual y afecta gravemente su calidad de vida [\cite{Landsend2021}.


No obstante, la aniridia no es solo un trastorno ocular, sino que también está
vinculada a diversas patologías sistémicas, lo que sugiere que sus mecanismos
genéticos y funcionales son complejos y multifactoriales\cite{BLACK2022389}. Por ejemplo, estudios
han mostrado que esta enfermedad puede estar asociada con el síndrome de
WAGR, una condición genética que incluye anomalías renales, tumores y problemas de desarrollo, además de la aniridia \cite{lopezrelacion}. Otros síndromes relacionados incluyen el síndrome de Gillespie, caracterizado por discapacidad intelectual, y el síndrome de Axenfeld-Rieger, que afecta tanto los ojos como otros órganos \cite{Law2011}. Estas asociaciones con condiciones sistémicas destacan la importancia de investigar en profundidad los mecanismos genéticos que subyacen
a la aniridia, así como su relación con otras patologías.


Uno de los avances más significativos en la investigación de la aniridia ha sido la 
identificación del gen \textbf{PAX6} como regulador clave del desarrollo ocular. PAX6 es un factor de transcripción que juega un papel crucial en
la diferenciación de diversas estructuras oculares, regulando la expresión de otros genes esenciales para este proceso \cite{robles_lopez_2012}. Las mutaciones en PAX6 han sido vinculadas
no solo con la aniridia, sino también con otras anomalías oculares como la displasia corneal (alteración en el desarrollo o estructura de la córnea) y el glaucoma
congénito  \cite{CalvaoPires2014}. Sin embargo, para entender completamente la heterogeneidad clínica de la aniridia, es fundamental analizar cómo interactúa
PAX6 con otros genes en redes genéticas más amplias.


Además de PAX6, varios genes han demostrado ser importantes para entender la
complejidad de la aniridia y sus trastornos asociados, entre ellos \textbf{FOXC1}, \textbf{WT1},
\textbf{COL4A1} y \textbf{PITX2}.

Las mutaciones en \textbf{FOXC1} \cite{NCBIGeneFOXC1}  están asociadas con el síndrome de Axenfeld-Rieger,
un trastorno que incluye glaucoma y otras anomalías oculares \cite{Reis2023}. Este gen codifica un factor de transcripción que, al igual que PAX6, regula
el desarrollo del ojo, y su interacción con otros genes es clave para comprender las
manifestaciones clínicas variadas que se presentan en estos pacientes. La integración
de FOXC1 en los análisis de redes génicas permite explorar cómo las vías de desarrollo ocular pueden estar interconectadas con otras rutas reguladoras sistémica.


Otro gen relevante es \textbf{WT1}, conocido por su papel en el desarrollo renal y la formación de tumores. Las mutaciones en WT1 están vinculadas al síndrome de
WAGR. La investigación sobre WT1 desde una perspectiva de biología de sistemas permite investigar cómo los defectos en los programas de desarrollo que afectan
tanto a los ojos como a otros órganos pueden estar mediadas por la interacción de redes génicas compartidas \cite{Pelletier1991}.


\textbf{COL4A1}, que codifica una cadena del colágeno tipo IV, es esencial para la integridad de las membranas basales en diversas estructuras del cuerpo, incluyendo
el ojo\cite{Vahedi2011}. Desde una perspectiva sistémica, las alteraciones en proteínas estructurales como el colágeno tipo IV pueden tener efectos pleiotrópicos (una mutación afecta a múltiples funciones del organismo), afectando no solo la morfogénesis ocular, sino también otras estructuras corporales dependientes de la integridad de las membranas basales.


Finalmente, el gen \textbf{PITX2} está involucrado en la morfogénesis craneofacial y cardiovascular. Las mutaciones en este gen están asociadas con el sóndrome de Axenfeld-Rieger, afectando no solo a los ojos, sino también al desarrollo de órganos internos, debido a su rol en la simetría del desarrollo\cite{French2021}. Desde la
perspectiva de la biología de sistemas, PITX2 representa un nodo crítico dentro de redes regulatorias que conectan el desarrollo de múltiples órganos, lo que resalta cómo las disfunciones en un solo gen pueden dar lugar a manifestaciones clínicas en varios sistemas.



El objetivo de este trabajo es examinar las patologías asociadas al fenotipo de la aniridia desde el punto de vista de la biología de sistemas, enfocándonos en los genes que tienen una relación funcional con \textbf{PAX6}. A través de la identificación de genes asociados y el análisis de redes génicas y moleculares mediante técnicas como la creación de clusters\cite{ben1999clustering} y el análisis de redes funcionales \cite{FloresCamacho2009}, se tratará de avanzar en la comprensión de cómo las interacciones entre estos genes contribuyen a la presentación clínica de la aniridia
y sus trastornos relacionados. Este enfoque no solo amplía nuestro conocimiento sobre la aniridia, sino que también ofrece información valiosa sobre los mecanismos
patológicos que interconectan diversas condiciones sistémicas.

