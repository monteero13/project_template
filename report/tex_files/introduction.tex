\section{Introducción}


La aniridia es una condición que se define como la parcial o completa hipoplasia del iris, el anillo coloreado que se sitúa alrededor de la pupila dentro del ojo. Este fenotipo puede ser ocasionado por un traumatismo, una cirugía complicada o un defecto genético. 
Este fenotipo se presenta como resultado de una mutación que impide el correcto desarrollo del globo ocular durante las primeras semanas de gestación. La aniridia se presenta como una enfermedad congénita aislada o asociada a otras patologías como el síndrome de WAGR, el tumor de Wilms, el síndrome de Gillespie o el síndrome de Axenfeld-Rieger.

\subsection{Información sobre los genes a estudiar}

La aniridia tiene una estrecha relación con el gen PAX-6, aunque no es el único gen asociado. A continuación comentaremos algunos de estos genes: 

\begin{itemize}
	\item PAX-6: este es un gen esencial debido a que codifica una proteína que actúa como un factor de transcripción, regulando la expresión de otros genes importantes en el desarrollo de diversas estructuras oculares como el iris.
	\item FOXC1: este gen es un factor de transcripción y, aunque aún no se ha determinado la función específica de este gen, se ha demostrado que desempeña un papel en la regulación del desarrollo embrionario y ocular. Las mutaciones en este gen causan varios fenotipos de glaucoma, incluido el glaucoma congénito primario, la anomalía de iridogoniodisgenesia autosómica dominante y la anomalía de Axenfeld-Rieger.
	\item Gen WT1: codifica un factor de transcripción, juega un papel crucial en el crecimiento y la diferenciación celular. Está altamente expresado en la leucemia y varios tipos de tumores sólidos, y se utiliza como marcador tumoral para detectar la enfermedad residual mínima en la leucemia.\ref{10.1093/jjco/hyp194}. Dependiendo del tipo de tumor, las proteínas WT1 pueden actuar como proteínas supresoras de tumores o como factores de supervivencia.\ref{SCHARNHORST2001141}
	\item COL4A1: codifica la cadena α1 del colágeno tipo IV, que es un componente crucial de casi todas las membranas basales, incluyendo aquellas en los vasos sanguíneos, los glomérulos renales y las estructuras oculares.\ref{Vahedi2011} Las mutaciones en COL4A1 pueden contribuir a anomalías oculares que afectan el desarrollo del segmento anterior, como la disgenesia del iris y otras estructuras oculares que también están alteradas en la aniridia.\ref{Reis2011}	
	
\end{itemize}


