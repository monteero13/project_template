\section{Introducción}

La aniridia es un trastorno congénito del desarrollo ocular caracterizado por la ausencia parcial o total del iris, la estructura que da color al ojo y que ayuda a regular la cantidad de luz que entra en la pupila. Esta condición no solo afecta la estética del ojo, sino que también puede conllevar una serie de problemas de visión, como fotofobia (sensibilidad a la luz), nistagmo (movimientos involuntarios del ojo) y una disminución de la agudeza visual. Los individuos con aniridia a menudo presentan malformaciones oculares asociadas, como cataratas, glaucoma y problemas en la córnea, lo que puede complicar aún más su salud visual y su calidad de vida \cite{Landsend2021}.

Además del compromiso ocular, la aniridia se asocia con múltiples patologías sistémicas, lo que sugiere que los mecanismos genéticos y funcionales detrás de esta enfermedad son complejos y multifactoriales. \cite{BLACK2022389} La investigación ha demostrado que la aniridia puede estar vinculada a síndromes como el síndrome de WAGR (Wilms tumor, Aniridia, Genitourinary abnormalities, and Range of developmental delays), que incluye no solo la aniridia, sino también anomalías renales y problemas de desarrollo \cite{lopezrelacion}. Otros síndromes asociados son el síndrome de Gillespie, que involucra aniridia y discapacidad mental, y el síndrome de Axenfeld-Rieger, que afecta a los ojos y otros órganos \cite{Law2011}. Estas asociaciones resaltan la importancia de investigar los mecanismos genéticos subyacentes a la aniridia y su relación con otras condiciones.

El gen principal asociado a la aniridia es el PAX6, que desempeña un papel crucial en el desarrollo y la diferenciación de los tejidos oculares. \cite{robles_lopez_2012} PAX6 es un factor de transcripción que regula la expresión de otros genes necesarios para la formación de diversas estructuras oculares. Las mutaciones en este gen han demostrado estar vinculadas no solo a la aniridia, sino también a otras anomalías oculares, como la displasia corneal y el glaucoma congénito \cite{CalvaoPires2014}. Sin embargo, la comprensión completa de cómo PAX6 interactúa con otros genes y qué otros genes están involucrados en la presentación clínica de la aniridia sigue siendo un área de estudio activa.

Además de PAX6, se han identificado otros genes relevantes en la patología de la aniridia, como FOXC1, WT1, COL4A1 y PITX2. Por ejemplo, las mutaciones en FOXC1 se han asociado con glaucoma y otras anomalías oculares en el contexto del síndrome de Axenfeld-Rieger \cite{Reis2023}. El gen WT1 es conocido por su papel en el desarrollo del riñón y en la formación de tumores, además de estar implicado en el desarrollo de anomalías oculares \cite{Pelletier1991}. COL4A1, que codifica una cadena del colágeno tipo IV, es esencial para la integridad de las membranas basales y se ha relacionado con varias anomalías oculares que afectan el desarrollo del segmento anterior \cite{Vahedi2011}. Finalmente, PITX2 es fundamental en la morfogénesis cardiovascular y en el desarrollo craneofacial, con mutaciones asociadas a trastornos que afectan la simetría y el desarrollo de órganos \cite{French2021}.

Este trabajo de investigación tiene como objetivo principal examinar las patologías asociadas al fenotipo de aniridia, centrándose en los genes que tienen una relación funcional con el gen PAX6. Mediante la identificación de genes relacionados, la creación de clusters \cite{ben1999clustering} y el análisis de sus mecanismos funcionales \cite{FloresCamacho2009}, buscamos profundizar en el entendimiento de cómo las alteraciones en estas interacciones génicas contribuyen al fenotipo de la aniridia y a otras enfermedades. Este enfoque no solo ampliará el conocimiento sobre la aniridia, sino que también proporcionará información valiosa sobre la interconexión de diversas condiciones patológicas y los mecanismos subyacentes que las vinculan.