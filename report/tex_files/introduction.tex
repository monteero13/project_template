La aniridia es un trastorno congénito del desarrollo ocular caracterizado por la ausencia parcial o total del iris, la estructura que da color al ojo y regula la cantidad de luz que entra en la pupila. Además de afectar la estética ocular, esta condición puede causar problemas visuales significativos, como fotofobia, nistagmo y disminución de la agudeza visual. Los individuos con aniridia a menudo presentan malformaciones asociadas, como cataratas, glaucoma y problemas en la córnea, lo que agrava su salud visual y afecta su calidad de vida \cite{Landsend2021}.

Sin embargo, la aniridia no es solo un trastorno ocular, sino que también está vinculada a diversas patologías sistémicas, lo que sugiere que sus mecanismos genéticos y funcionales son complejos. Por ejemplo, estudios han mostrado que esta enfermedad está asociada con el síndrome de WAGR, que incluye anomalías renales, tumores y problemas de desarrollo, además de la aniridia \cite{lopezrelacion}. Otros síndromes relacionados incluyen el síndrome de Gillespie, caracterizado por discapacidad intelectual, y el síndrome de Axenfeld-Rieger, que afecta tanto los ojos como otros órganos \cite{Law2011}. Estas asociaciones destacan la importancia de investigar los mecanismos genéticos de la aniridia y su relación con otras patologías.

Uno de los avances más importantes en la investigación de la aniridia ha sido la identificación del gen \textbf{PAX6} como regulador clave del desarrollo ocular. PAX6 es un factor de transcripción crucial para la diferenciación de diversas estructuras oculares, regulando la expresión de otros genes esenciales para este proceso \cite{robles_lopez_2012}. Las mutaciones en PAX6 se han vinculado no solo con la aniridia, sino también con otras anomalías oculares como la displasia corneal y el glaucoma congénito \cite{CalvaoPires2014}. Sin embargo, para comprender completamente la heterogeneidad clínica de la aniridia, es necesario estudiar cómo PAX6 interactúa con otros genes en redes genéticas más amplias.

A través de un enfoque de biología de sistemas, en este trabajo se exploran las interacciones entre \textbf{PAX6} y otros genes asociados con la aniridia mediante el análisis de redes genéticas y moleculares. Utilizando el algoritmo DIAMOnD, se propaga una red basada en la especie \textit{Homo sapiens} descargada de STRINGdb, y se identifican genes relevantes en la red utilizando métricas como el índice de betweenness y el grado de centralidad. Estos genes se agrupan mediante técnicas de clusterización y se realiza un análisis de enriquecimiento de los clusters obtenidos. Este enfoque nos permite analizar las interacciones entre genes que podrían contribuir a la presentación clínica de la aniridia y sus trastornos relacionados.

Además, se exploran los fenotipos asociados con los genes enriquecidos, estableciendo posibles conexiones entre las enfermedades y los fenotipos mediante la integración de datos de un archivo adicional de fenotipos. Al asociar estos fenotipos con genes y enfermedades, este trabajo no solo amplía nuestro entendimiento sobre la aniridia, sino que también ofrece una visión integral sobre los mecanismos patológicos subyacentes a varias condiciones sistémicas relacionadas con esta enfermedad.

Este análisis tiene como objetivo avanzar en la comprensión de cómo las redes genéticas y sus interacciones funcionales pueden explicar la heterogeneidad clínica de la aniridia y sus trastornos asociados, proporcionando nuevas perspectivas sobre los mecanismos que vinculan diversos fenotipos y enfermedades.