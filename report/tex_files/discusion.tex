\section{Discusión}

El análisis realizado a lo largo de este artículo ofrece un enfoque integral para el estudio de la \textbf{aniridia}, destacando la relevancia de la integración de herramientas de bioinformática y bases de datos especializadas en fenotipos humanos como HPO (Human Phenotype Ontology) para la identificación y caracterización de genes asociados. A partir de un fenotipo inicial relacionado con la aniridia, el estudio utilizó una red proteica generada a través de StringDB para identificar genes clave y extender la red a través de propagación. Este enfoque permitió la construcción de una red de 200 nodos, centrado en los genes semilla WNT10B, WT1, SEM1, PAX6, y NF1, los cuales son biológicamente relevantes para los mecanismos subyacentes de la enfermedad.

La selección de estos genes semilla resultó fundamental para orientar el análisis posterior de la red, ya que su implicación en procesos relacionados con el desarrollo ocular y los fenotipos asociados a la aniridia ya había sido descrita previamente en la literatura. La propagación de la red con base en estos genes permitió capturar interacciones adicionales y posibles moduladores que podrían estar implicados en la etiología del fenotipo estudiado. Este enfoque asegura no solo la identificación de elementos centrales en la red, sino también de genes secundarios o periféricos que podrían influir en los procesos moleculares relacionados.

El clustering de la red, una etapa crítica del análisis, permitió agrupar genes en comunidades funcionales que reflejan su co-implicación en procesos biológicos específicos. El enriquecimiento funcional de los clusters identificados reveló que al menos la mitad de los fenotipos asociados a estos grupos están directamente relacionados con los fenotipos de la aniridia descritos en HPO. Este resultado refuerza la robustez del enfoque metodológico, ya que confirma la coherencia entre los datos derivados de la red y el conocimiento previo. Además, destaca la capacidad de las técnicas de clustering para identificar módulos funcionales con una alta relevancia biológica.

Los genes semilla identificados, en particular PAX6, un gen base en el desarrollo ocular, y WT1, relacionado con procesos de desarrollo y cáncer, resaltan la complejidad de las interacciones genéticas y moleculares que subyacen a la aniridia. La identificación de genes adicionales en la red ampliada sugiere que el fenotipo no solo está condicionado por estos genes principales, sino también por redes más amplias de interacciones genéticas. Estos hallazgos abren nuevas posibilidades para explorar terapias dirigidas o identificar biomarcadores adicionales para diagnóstico y pronóstico.

El enriquecimiento funcional subraya la relación entre los procesos moleculares y los fenotipos clínicos observados en pacientes con aniridia. Esta conexión entre fenotipo y genotipo tiene aplicaciones prácticas en genética clínica, ya que permite refinar los diagnósticos y mejorar las predicciones relacionadas con manifestaciones fenotípicas adicionales. Por ejemplo, la implicación de genes relacionados con el desarrollo del sistema nervioso y el sistema inmunológico podría explicar fenómenos clínicos más allá de los defectos oculares.

Aunque el análisis de la red y el clustering resultaron útiles para identificar genes y procesos asociados, existen limitaciones inherentes al uso de bases de datos como StringDB y HPO. Las redes generadas dependen de los datos disponibles y pueden estar sesgadas hacia interacciones conocidas, subestimando las interacciones novedosas o contextuales. Además, el enriquecimiento funcional se basa en anotaciones existentes, por lo que la interpretación de resultados depende de la cobertura y calidad de estas anotaciones.

Futuros estudios podrían ampliar el análisis utilizando otras metodologías complementarias, como datos transcriptómicos o epigenómicos de tejidos afectados por la aniridia. También sería valioso realizar validaciones experimentales para confirmar las predicciones de la red y explorar los mecanismos moleculares subyacentes en modelos animales o líneas celulares. Finalmente, la integración de datos clínicos específicos de pacientes podría refinar aún más el análisis, permitiendo una personalización del diagnóstico y el tratamiento.