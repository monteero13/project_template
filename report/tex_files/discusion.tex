\section{Discusión}

El análisis de la \textbf{aniridia} como fenotipo central permitió identificar múltiples asociaciones relevantes con enfermedades genéticas descritas en bases de datos especializadas como OMIM y Orphanet. En los datos analizados, se observa que la aniridia (HPO:0000526) está presente en una amplia variedad de condiciones, reflejando su relevancia clínica y biológica. Estas incluyen, entre otras, *Axenfeld-Rieger syndrome, type 1* (OMIM:180500), *D-lactic aciduria with susceptibility to gout* (OMIM:245450), *Nephroblastoma* (ORPHA:654) y *Chromosome 11p13 deletion syndrome, distal* (OMIM:616902). Cada una de estas enfermedades aporta perspectivas complementarias sobre los mecanismos genéticos y moleculares asociados a la aniridia.

El gen \textit{PAX6}, ampliamente reconocido como un regulador maestro en el desarrollo ocular, surge como un elemento central en el análisis de redes de interacción genética. Este gen no solo está directamente relacionado con la aniridia, sino que también actúa como un nodo crítico en las redes moleculares que controlan procesos como la diferenciación de tejidos oculares, el desarrollo neuronal y la señalización celular. Adicionalmente, genes como \textit{WT1} y \textit{NF1}, identificados en las redes extendidas generadas, refuerzan la idea de que la aniridia no es un fenotipo aislado, sino que forma parte de una red compleja de interacciones genéticas con implicaciones multisistémicas.

La diversidad de enfermedades asociadas a la aniridia destaca la importancia de abordar este fenotipo desde una perspectiva interdisciplinaria. Por ejemplo, en condiciones como el *Nephroblastoma* y el *Chromosome 11p13 deletion syndrome*, la aniridia es acompañada de fenotipos adicionales como hipertensión, hemihipertrofia y retraso en el desarrollo psicomotor. Estas asociaciones subrayan la necesidad de considerar no solo los fenotipos primarios, sino también los secundarios o relacionados, para realizar diagnósticos más precisos.

El análisis funcional de los fenotipos asociados también resalta que un número significativo de enfermedades relacionadas con la aniridia involucran procesos moleculares como la regulación del desarrollo embrionario, la señalización del factor de crecimiento transformante beta (TGF-β), y la reparación del ADN. Esto sugiere que los mecanismos subyacentes a la aniridia podrían incluir disfunciones en rutas moleculares comunes que contribuyen tanto a defectos estructurales como a susceptibilidades a enfermedades malignas.

Aunque el presente análisis se centra en asociaciones conocidas y documentadas en bases de datos como HPO, OMIM y Orphanet, es fundamental reconocer las limitaciones de estas fuentes. Las interacciones gen-fenotipo podrían estar infraestimadas debido a la falta de datos en poblaciones específicas o por la subrepresentación de variantes genéticas raras en estas bases de datos. Por lo tanto, futuros estudios deberían incluir enfoques complementarios, como el análisis transcriptómico y proteómico en tejidos afectados, para identificar patrones moleculares específicos de la aniridia.

En conclusión, la \textbf{aniridia} no solo es un marcador clínico de enfermedades genéticas específicas, sino también una ventana a la comprensión de redes moleculares más amplias que subyacen a defectos del desarrollo y condiciones multisistémicas. La integración de herramientas bioinformáticas, redes moleculares y datos clínicos puede mejorar significativamente el diagnóstico y manejo de estas enfermedades, abriendo además oportunidades para desarrollar terapias dirigidas basadas en los mecanismos subyacentes identificados.
